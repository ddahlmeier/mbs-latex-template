% ------------------------------------------------------------------------
%
% Variables
%
% ------------------------------------------------------------------------
\newcommand{\EMBALecturer}{Prof LecturerFirstName LecturerLastName}
\newcommand{\EMBAAssignmentType}{Type of Assignment}
\newcommand{\EMBAParticipant}{ParticipantFirstName ParticipantLastName}
\newcommand{\EMBAClass}{Mannheim Executive MBA 2019}
\newcommand{\EMBACourse}{Name of the Course}
\newcommand{\EMBALocation}{Mannheim}


% ------------------------------------------------------------------------
%
% Preamble
%
% ------------------------------------------------------------------------
\documentclass[
    12pt                               % Font size 12 points
    ]{article}

\usepackage[utf8]{inputenc}            % Accept German characters
\usepackage{lmodern}                   % High quality fonts
\usepackage[english]{babel}            % English word breaks
\usepackage{indentfirst}               % intent first paragraph
\usepackage{times}                     % Times New Roman font
\usepackage{courier}                   % Monospace for inline fixed text with \texttt
\usepackage{natbib}                    % Chicago style references
\usepackage[onehalfspacing]{setspace}  % Line spacing
\usepackage{acro}                      % Abbreviations

\usepackage[pdftex,
            pdfauthor=,                % Do NOT specify author!
            pdftitle=\EMBACourse,      % PDF Title
            pdfsubject=\EMBAClass,     % PDF Subject
            hidelinks]                 % Hyperlinks
            {hyperref}

\usepackage{geometry}                  % Page margins
\geometry{
  a4paper,
  left=25mm,
  right=25mm
}

\author{}                             % Do NOT specify author!



% ------------------------------------------------------------------------
%
% Acronyms
%
% ------------------------------------------------------------------------
\DeclareAcronym{ny}{
  short = NY ,
  long  = New York ,
  class = abbrev
}


% ------------------------------------------------------------------------
%
% Lead
%
% ------------------------------------------------------------------------

\title{Assignment Title}

\begin{document}  

\maketitle

% set paragraph indent to 5 spaces
\settowidth{\parindent}{~~~~~}

\tableofcontents
\listoffigures
\listoftables


% ------------------------------------------------------------------------
%
% Content
%
% ------------------------------------------------------------------------

\section{Basic Commands}

German characters, such as Ä und ö, can only be typed directly if you
make use of the \texttt{texttt} package.

You reference acronyms, such as \ac{ny}, using the \texttt{ac} command.
The first time you use the acronym, it appears in full. From that point
onwards it will only be rendered in compact form.


\section{Citations}

The Chicago bitation has two styles: within the text they appear in 
parentheses as \citep{zorlu2005effect} or, if the author’s name appears
in the text itself, as \citet{zorlu2005effect}.


% ------------------------------------------------------------------------
%
% Bibliography
%
% ------------------------------------------------------------------------

\bibliographystyle{chicago}
\bibliography{references}


% ------------------------------------------------------------------------
%
% Appendix
%
% ------------------------------------------------------------------------

\appendix
\section{Reference Material}

Write content as normally here.


\section{Use Cases}

Even more reference material.


% ------------------------------------------------------------------------
%
% Acronyms
%
% ------------------------------------------------------------------------

\printacronyms[include-classes=abbrev,name=Abbreviations]

\end{document}
% eof
